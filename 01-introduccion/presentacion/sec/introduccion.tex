\section{Introducción}
\subsection{Instalación}
\begin{frame}[fragile]
  \frametitle{Introducción}
  \framesubtitle{Instalación}
  \begin{itemize}
  \item\texttt{Con gestores de paquetes}
    \begin{itemize}
    \item\texttt{Debian/Ubuntu}
      {\color{white}
        \inputminted[bgcolor=bg]{text}{code/instalacion01.txt}
      }
    \item\texttt{Arch}
      {\color{white}
        \inputminted[bgcolor=bg]{text}{code/instalacion02.txt}
      }
    \item\texttt{macOS}
      {\color{white}
          \inputminted[bgcolor=bg]{text}{code/instalacion03.txt}
      }
    \end{itemize}
  \item\texttt{Desde la web}\\
    \url{https://discourse.acmupm.es/t/curso-de-introduccion-a-la-programacion-funcional/296}
    % \url{https://www.haskell.org/platform/mac.html}\\
    % \url{https://www.haskell.org/platform/windows.html}\\
    % \url{https://www.haskell.org/downloads/linux}
  \end{itemize}
\end{frame}

\subsection{Introducción Histórica}
\begin{frame}[fragile]          %TODO
  \frametitle{Introducción}
  \framesubtitle{Introducción Histórica: \textit{The Haskell Journey} - Simon Peyton Jones}
  \url{https://goo.gl/qyPpMk}
\end{frame}

\subsection{¿Qué es Haskell?}
\begin{frame}{Intoducción}      %TODO
  \framesubtitle{¿Qué es Haskell?}
\end{frame}
