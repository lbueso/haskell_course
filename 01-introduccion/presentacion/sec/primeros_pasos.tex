\section{Primeros Pasos}
\subsection{GHCi}
\begin{frame}[fragile]
  \frametitle{Primeros pasos}
  \framesubtitle{GHCi}
  \begin{itemize}
  \item\texttt{Así abrimos el intérprete de Haskell}
    {\color{white}
      \begin{minted}[bgcolor=bg]{text}
  > ghci
  GHCi, version 8.0.2: http://www.haskell.org/ghc/
  :? for help
  Prelude>
      \end{minted}
    }

  \item\texttt{Podemos escribir expresiones aritméticas y lógicas:}
    {\color{white}
      \begin{minted}[bgcolor=bg]{text}
  Prelude> 2 + 2
  4
  Prelude> True && False
  False
      \end{minted}
    }
  \end{itemize}
\end{frame}

\begin{frame}[fragile]
  \frametitle{Primeros pasos}
  \framesubtitle{GHCi}
  \begin{itemize}
  \item\texttt{Podemos llamar a funciones}
    {\color{white}
      \begin{minted}[bgcolor=bg]{text}
  Prelude> 2 + 2
  4
  Prelude> True && False
  False
      \end{minted}
    }

  \item\texttt{Errores}
    {\color{white}
      \begin{minted}[bgcolor=bg]{text}
  Prelude> 2 + "hola"

  <interactive>:6:1: error:
  • No instance for (Num [Char]) arising from a use
  of ‘+’
  • In the expression: 2 + "hola"
  In an equation for ‘it’: it = 2 + "hola"
      \end{minted}
    }
  \end{itemize}
\end{frame}

\begin{frame}[fragile]
  \frametitle{Primeros pasos}
  \framesubtitle{GHCi}
  \begin{itemize}
  \item\texttt{Expresiones útiles:}
    {\color{white}
      \begin{minted}[bgcolor=bg]{text}
   Prelude> :t 5
   5 :: Num t => t
   Prelude> :t 2
   2 :: Num t => t
   Prelude> :t "hola"
   "hola" :: [Char]

   Prelude> :l introduccion.hs
   [1 of 1] Compiling Main
   ( introduccion.hs, interpreted )
   Ok, modules loaded: Main.
   *Main>
      \end{minted}
    }
  \end{itemize}
\end{frame}

\subsection{Primeras funciones}
\begin{frame}[fragile]
  \frametitle{Primeros pasos}
  \framesubtitle{Primeras funciones}
  % \begin{block}{Función:}

  % \end{block}
  \texttt{Ejemplos:}
  {\color{white}
    \begin{minted}[bgcolor=bg]{haskell}
  doubleMe x = x + x

  doubleUs x y = doubleMe x + doubleMe y

  doubleSmallNumber x = (if x > 100 then x else x*2) + 1
    \end{minted}
  }
\end{frame}

\subsection{Listas}
\begin{frame}[fragile]
  \frametitle{Primeros pasos}
  \framesubtitle{Listas}
  % \begin{block}{Listas:}
  % \end{block}
  \texttt{Creando listas en un fichero \verb~.hs~:}
  {\color{white}
    \begin{minted}[bgcolor=bg]{haskell}
  list = [1,2,3,4,5]
  list' = [1..5]
    \end{minted}
  }
  \texttt{Ejemplos de listas en el GHCi}
  {\color{white}
    \begin{minted}[bgcolor=bg]{text}
  Prelude> let list = [1,2,3,4,5]
  Prelude> let list1 = [1..5]
  Prelude> list == list1
  list == list1
  True
  Prelude>
    \end{minted}
  }
\end{frame}

\begin{frame}[fragile]
  \frametitle{Primeros pasos}
  \framesubtitle{Listas}
  Las listas pueden concatenarse con el operador \verb~++~ y\\construirse con el operador \verb~:~
  {\color{white}
    \begin{minted}[bgcolor=bg]{text}
  Prelude> [1..9] ++ [10..19]
  [1,2,3,4,5,6,7,8,9,10,11,12,13,14,15,16,17,18,19]
  Prelude> "hello " ++ "world!"
  "hello world!"
  Prelude> 1:[2..5]
  1:[2..5]
  [1,2,3,4,5]
  Prelude> 'h':"ello"
  "hello"
  Prelude> 1:text [1]
    \end{minted}
  }
\end{frame}

\begin{frame}[fragile]
  \frametitle{Primeros pasos}
  \framesubtitle{Listas}
  \texttt{Las operaciones básicas en las listas son:}
  \begin{itemize}
  \item \verb~head~:
    {\color{white}
      \begin{minted}[bgcolor=bg]{text}
  Prelude> head [1..10]
  1
      \end{minted}
    }
  \item \verb~tail~:
    {\color{white}
      \begin{minted}[bgcolor=bg]{text}
  Prelude> tail [1..10]
  [2,3,4,5,6,7,8,9,10]
    \end{minted}
    }
  \item \verb~last~:
    {\color{white}
      \begin{minted}[bgcolor=bg]{text}
  Prelude> last [1..10]
  10
    \end{minted}
    }
  \end{itemize}
\end{frame}

\begin{frame}[fragile]
  \frametitle{Primeros pasos}
  \framesubtitle{Listas}
  \begin{itemize}
  \item \verb~init~:
    {\color{white}
      \begin{minted}[bgcolor=bg]{text}
  Prelude> init [1..10]
  [1,2,3,4,5,6,7,8,9]
    \end{minted}
    }
  \item \verb~length~:
    {\color{white}
      \begin{minted}[bgcolor=bg]{text}
  Prelude> length [1..10]
  10
    \end{minted}
    }
  \item \verb~reverse~:
    {\color{white}
      \begin{minted}[bgcolor=bg]{text}
  Prelude> reverse [1..10]
  [10,9,8,7,6,5,4,3,2,1]
    \end{minted}
    }
  \item \verb~take~:
    {\color{white}
      \begin{minted}[bgcolor=bg]{text}
   Prelude> take 4 [1..10]
   [1,2,3,4]
    \end{minted}
    }
  \end{itemize}
\end{frame}

\subsection{Tuplas}
\begin{frame}[fragile]
  \frametitle{Primeros pasos}
  \framesubtitle{Tuplas}
  {\color{white}
    \begin{minted}[bgcolor=bg]{haskell}
  tuple = (1, False)
  \end{minted}
  }
  \texttt{Las operaciones básicas en las tuplas son:}
  \begin{itemize}
  \item \verb~fst~
    {\color{white}
      \begin{minted}[bgcolor=bg]{text}
  Prelude> let tuple = (1, False)
  Prelude> fst tuple
  1
    \end{minted}
    }
  \item \verb~snd~
    {\color{white}
      \begin{minted}[bgcolor=bg]{text}
  Prelude> let tuple = (1, False)
  Prelude> snd tuple
  False
  Prelude>
    \end{minted}
    }
  \end{itemize}
\end{frame}
