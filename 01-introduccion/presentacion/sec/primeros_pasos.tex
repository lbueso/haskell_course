\section{Primeros Pasos}
\subsection{GHCi}
\begin{frame}[fragile]
  \frametitle{Primeros pasos}
  \framesubtitle{GHCi}
  \begin{itemize}
  \item\texttt{Así abrimos el intérprete de Haskell}
    {\color{white}
      \inputminted[bgcolor=bg]{text}{code/primeros_pasos01.txt}
    }

  \item\texttt{Podemos escribir expresiones aritméticas y lógicas:}
    {\color{white}
      \inputminted[bgcolor=bg]{text}{code/primeros_pasos02.txt}
    }
  \end{itemize}
\end{frame}

\begin{frame}[fragile]
  \frametitle{Primeros pasos}
  \framesubtitle{GHCi}
  \begin{itemize}
  \item\texttt{Podemos llamar a funciones}
    {\color{white}
      \inputminted[bgcolor=bg]{text}{code/primeros_pasos03.txt}
    }

  \item\texttt{Errores}
    {\color{white}
        \inputminted[bgcolor=bg]{text}{code/primeros_pasos04.txt}
    }
  \end{itemize}
\end{frame}

\begin{frame}[fragile]
  \frametitle{Primeros pasos}
  \framesubtitle{GHCi}
  \begin{itemize}
  \item\texttt{Expresiones útiles:}
    {\color{white}
      \inputminted[bgcolor=bg]{text}{code/primeros_pasos05.txt}
    }
  \end{itemize}
\end{frame}

\subsection{Primeras funciones}
\begin{frame}[fragile]
  \frametitle{Primeros pasos}
  \framesubtitle{Primeras funciones}
  % \begin{block}{Función:}

  % \end{block}
  \texttt{Ejemplos:}
  {\color{white}
    \inputminted[bgcolor=bg]{haskell}{code/primeras_funciones01.hs}
  }
\end{frame}

\subsection{Listas}
\begin{frame}[fragile]
  \frametitle{Primeros pasos}
  \framesubtitle{Listas}
  % \begin{block}{Listas:}
  % \end{block}
  \texttt{Creando listas en un fichero \verb~.hs~:}
  {\color{white}
    \inputminted[bgcolor=bg]{haskell}{code/primeras_funciones02.hs}
  }
  \texttt{Ejemplos de listas en el GHCi}
  {\color{white}
    \inputminted[bgcolor=bg]{text}{code/primeras_funciones03.txt}
  }
\end{frame}

\begin{frame}[fragile]
  \frametitle{Primeros pasos}
  \framesubtitle{Listas}
  Las listas pueden concatenarse con el operador \verb~++~ y\\construirse con el operador \verb~:~
  {\color{white}
    \inputminted[bgcolor=bg]{text}{code/primeras_funciones04.txt}
  }
\end{frame}

\begin{frame}[fragile]
  \frametitle{Primeros pasos}
  \framesubtitle{Listas}
  \texttt{Las operaciones básicas en las listas son:}
  \begin{itemize}
  \item \verb~head~:
    {\color{white}
      \inputminted[bgcolor=bg]{text}{code/primeras_funciones05.txt}
    }
  \item \verb~tail~:
    {\color{white}
      \inputminted[bgcolor=bg]{text}{code/primeras_funciones06.txt}
    }
  \item \verb~last~:
    {\color{white}
      \inputminted[bgcolor=bg]{text}{code/primeras_funciones07.txt}
    }
  \end{itemize}
\end{frame}

\begin{frame}[fragile]
  \frametitle{Primeros pasos}
  \framesubtitle{Listas}
  \begin{itemize}
  % \item \verb~init~:
  %   {\color{white}
  %     \inputminted[bgcolor=bg]{text}{code/primeras_funciones08.txt}
  %   }
  \item \verb~length~:
    {\color{white}
      \inputminted[bgcolor=bg]{text}{code/primeras_funciones09.txt}
    }
  \item \verb~reverse~:
    {\color{white}
      \inputminted[bgcolor=bg]{text}{code/primeras_funciones10.txt}
    }
  \item \verb~take~:
    {\color{white}
      \inputminted[bgcolor=bg]{text}{code/primeras_funciones11.txt}
    }
  \end{itemize}
\end{frame}

\subsection{Tuplas}
\begin{frame}[fragile]
  \frametitle{Primeros pasos}
  \framesubtitle{Tuplas}
  {\color{white}
    \inputminted[bgcolor=bg]{haskell}{code/tuplas01.hs}
  }
  \texttt{Las operaciones básicas en las tuplas son:}
  \begin{itemize}
  \item \verb~fst~
    {\color{white}
      \inputminted[bgcolor=bg]{text}{code/tuplas02.txt}
    }
  \item \verb~snd~
    {\color{white}
      \inputminted[bgcolor=bg]{text}{code/tuplas03.txt}
    }
  \end{itemize}
\end{frame}
