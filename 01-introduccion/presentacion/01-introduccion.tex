\documentclass{beamer}

\usetheme{simple}

\usepackage{lmodern}
\usepackage[scale=2]{ccicons}
\usepackage[utf8]{inputenc}
\usepackage[spanish]{babel}

% TODO:
% position adjustement
% change colours
% 

% Watermark background (simple theme)

\setwatermark{\includegraphics[height=8cm]{img/Haskell-Logo.png}}


\title{Introducción a Haskell}
\subtitle{}
\date{\today}
\author{Ignacio Ballesteros,\\Luis Eduardo Bueso,\\Anabel Ovide}
\institute{\url{https://github.com/edububa/haskell_course}}

\begin{document}

\maketitle

\begin{frame}{Introducción}
  \framesubtitle{Instalación}

  \texttt{Con gestores de paquetes}

  \begin{block}{Debian/Ubuntu}
% \begin{verbatim}
    sudo apt-get update\\
    sudo apt-get install ghc
% \end{verbatim}
  \end{block}

  \begin{block}{Arch}
% \begin{verbatim}
    sudo pacman -S ghc
% \end{verbatim}
  \end{block}

  \begin{block}{macOS}
% \begin{verbatim}
    brew install ghc
% \end{verbatim}
  \end{block}

  \texttt{Desde la web}\\
  https://www.haskell.org/platform/mac.html\\
  https://www.haskell.org/platform/windows.html\\
  https://www.haskell.org/downloads/linux

\end{frame}

\begin{frame}{Introducción}     %TODO
  \framesubtitle{Introducción Histórica}

\end{frame}

\begin{frame}{Intoducción}      %TODO
  \framesubtitle{¿Qué es Haskell?}

\end{frame}

\begin{frame}{Primeros pasos}      %TODO
  \framesubtitle{GHCi}

\end{frame}

\begin{frame}{Primeros pasos}      %TODO
  \framesubtitle{Primeras funciones}

\end{frame}

\begin{frame}{Primeros pasos}      %TODO
  \framesubtitle{Listas}

\end{frame}

\begin{frame}{Primeros pasos}      %TODO
  \framesubtitle{Tuplas}

\end{frame}

\begin{frame}{Tipos y \emph{Typeclasses}}      %TODO

\end{frame}

\begin{frame}{Sintaxis en funciones}      %TODO

\end{frame}

\begin{frame}{Recursión}      %TODO

\end{frame}

\end{document}
