\documentclass{beamer}

\usetheme{simple}

\usepackage{lmodern}
\usepackage[scale=2]{ccicons}
\usepackage[utf8]{inputenc}
\usepackage[spanish]{babel}
\usepackage{minted}

% TODO:
% position adjustement
% change colours
% 

% Watermark background (simple theme)

\setwatermark{\includegraphics[height=8cm]{img/Haskell-Logo.png}}

\title{Introducción a Haskell}
\subtitle{}
\date{\today}
\author{Ballesteros, Ignacio\\Bueso, Luis Eduardo\\Ovide Anabel}
\institute{\url{https://github.com/edububa/haskell_course}}

\begin{document}

\maketitle

\begin{frame}{Introducción}[fragile]
  \framesubtitle{Instalación}

  \texttt{Con gestores de paquetes}

  \begin{block}{Debian/Ubuntu}
    \begin{minted}{shell}
      sudo apt-get update\\
      sudo apt-get install ghc
    \end{minted}
  \end{block}

  \begin{block}{Arch}
    \begin{minted}{shell}
      sudo pacman -S ghc
    \end{minted}
  \end{block}

  \begin{block}{macOS}
    \begin{minted}{bash}
      brew install ghc
    \end{minted}
  \end{block}

  \texttt{Desde la web}\\
  https://www.haskell.org/platform/mac.html\\
  https://www.haskell.org/platform/windows.html\\
  https://www.haskell.org/downloads/linux

\end{frame}

\begin{frame}{Introducción}     %TODO
  \framesubtitle{Introducción Histórica}

\end{frame}

\begin{frame}{Intoducción}      %TODO
  \framesubtitle{¿Qué es Haskell?}

\end{frame}

% \begin{frame}{Primeros pasos}      %TODO
%   \framesubtitle{GHCi}

%   \texttt{Así abrimos el intérprete de Haskell}
%   \begin{block}
%     \begin{minted}{bash}
%       \$ ghci
%       GHCi, version 8.0.2: http://www.haskell.org/ghc/  :? for help
%       Prelude>
%     \end{minted}
%   \end{block}

%   \begin{block}{Podemos escribir expresiones aritméticas y lógicas:}
%     \begin{minted}{bash}
%       Prelude> 2 + 2
%       4
%       Prelude> True && False
%       False
%     \end{minted}
%   \end{block}

%   \begin{block}{Podemos llamar a funciones}
%     \begin{minted}{bash}
%       Prelude> 2 + 2
%       4
%       Prelude> True && False
%       False
%     \end{minted}
%   \end{block}

%   \begin{block}{Errores}
%     \begin{minted}{bash}
%       Prelude> 2 + "hola"

%       <interactive>:6:1: error:
%       • No instance for (Num [Char]) arising from a use of ‘+’
%       • In the expression: 2 + "hola"
%       In an equation for ‘it’: it = 2 + "hola"
%     \end{minted}
%   \end{block}

%   \begin{block}{Expresiones útiles}
%     \begin{minted}{bash}
%       Prelude> :t 5
%       5 :: Num t => t
%       Prelude> :t 2
%       2 :: Num t => t
%       Prelude> :t "hola"
%       "hola" :: [Char]

%       Prelude> :l introduccion.hs
%       [1 of 1] Compiling Main             ( introduccion.hs, interpreted )
%       Ok, modules loaded: Main.
%       *Main>
%     \end{minted}
%   \end{block}
% \end{frame}

% \begin{frame}{Primeros pasos}      %TODO
%   \framesubtitle{Primeras funciones}

% \end{frame}

% \begin{frame}{Primeros pasos}      %TODO
%   \framesubtitle{Listas}

% \end{frame}

% \begin{frame}{Primeros pasos}      %TODO
%   \framesubtitle{Tuplas}

% \end{frame}

% \begin{frame}{Tipos y \emph{Typeclasses}}      %TODO

% \end{frame}

% \begin{frame}{Sintaxis en funciones}      %TODO

% \end{frame}

% \begin{frame}{Recursión}      %TODO

% \end{frame}

\end{document}
