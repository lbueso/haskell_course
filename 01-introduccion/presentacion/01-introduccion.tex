\documentclass{beamer}

\usetheme{simple}

\usepackage{lmodern}
\usepackage[scale=2]{ccicons}
\usepackage[utf8]{inputenc}
\usepackage[spanish]{babel}
\usepackage{minted}

\usemintedstyle{emacs}

% TODO:
% position adjustement
% change colours
% 

% Watermark background (simple theme)

\setwatermark{\includegraphics[height=8cm]{img/Haskell-Logo.png}}

\title{Introducción a Haskell}
\subtitle{}
\date{\today}
\author{Ballesteros, Ignacio\\Bueso, Luis Eduardo\\Ovide, Anabel}
\institute{\url{https://github.com/edububa/haskell_course}}

\begin{document}

\maketitle

\frame{\frametitle{Índice}\tableofcontents}

\definecolor{bg}{rgb}{0.95,0.95,0.95}

%%%%%%%%%%%%%%%%%%%%%%%%%%%%%%%%%%%%%%%%%%%%%%%%%%%%%%%%%%%%%%%%%%%%%%%%%%%%% 
% INTRODUCCION
%%%%%%%%%%%%%%%%%%%%%%%%%%%%%%%%%%%%%%%%%%%%%%%%%%%%%%%%%%%%%%%%%%%%%%%%%%%%% 
\section{Introducción}
\subsection{Instalación}
\begin{frame}[fragile]
  \frametitle{Introducción}
  \framesubtitle{Instalación}
  \begin{itemize}
  \item\texttt{Con gestores de paquetes}
    \begin{itemize}
    \item\texttt{Debian/Ubuntu}
      \begin{minted}[bgcolor=bg]{shell}
        sudo apt-get update
        sudo apt-get install ghc
      \end{minted}
    \item\texttt{Arch}
      \begin{minted}[bgcolor=bg]{shell}
        sudo pacman -S ghc
      \end{minted}
    \item\texttt{macOS}
      \begin{minted}[bgcolor=bg]{shell}
        brew install ghc
      \end{minted}
    \end{itemize}
  \item\texttt{Desde la web}\\
    https://www.haskell.org/platform/mac.html\\
    https://www.haskell.org/platform/windows.html\\
    https://www.haskell.org/downloads/linux
  \end{itemize}
\end{frame}

\subsection{Introducción Histórica}
\begin{frame}[fragile]          %TODO
  \frametitle{Introducción}
  \framesubtitle{Introducción Histórica}

\end{frame}

\subsection{¿Qué es Haskell?}
\begin{frame}{Intoducción}      %TODO
  \framesubtitle{¿Qué es Haskell?}

\end{frame}

%%%%%%%%%%%%%%%%%%%%%%%%%%%%%%%%%%%%%%%%%%%%%%%%%%%%%%%%%%%%%%%%%%%%%%%%%%%%% 
% PRIMEROS PASOS
%%%%%%%%%%%%%%%%%%%%%%%%%%%%%%%%%%%%%%%%%%%%%%%%%%%%%%%%%%%%%%%%%%%%%%%%%%%%% 
\section{Primeros Pasos}
\subsection{GHCi}
\begin{frame}[fragile]
  \frametitle{Primeros pasos}
  \framesubtitle{GHCi}
  \begin{itemize}
  \item\texttt{Así abrimos el intérprete de Haskell}
    \begin{minted}[bgcolor=bg]{shell}
      \$ ghci
      GHCi, version 8.0.2: http://www.haskell.org/ghc/
      :? for help
      Prelude>
    \end{minted}

  \item\texttt{Podemos escribir expresiones aritméticas y lógicas:}
    \begin{minted}[bgcolor=bg]{shell}
      Prelude> 2 + 2
      4
      Prelude> True && False
      False
    \end{minted}
  \end{itemize}
\end{frame}

\begin{frame}[fragile]
  \frametitle{Primeros pasos}
  \framesubtitle{GHCi}
  \begin{itemize}
  \item\texttt{Podemos llamar a funciones}
    \begin{minted}[bgcolor=bg]{bash}
      Prelude> 2 + 2
      4
      Prelude> True && False
      False
    \end{minted}

  \item\texttt{Errores}
    \begin{minted}[bgcolor=bg]{shell}
      Prelude> 2 + "hola"

      <interactive>:6:1: error:
      • No instance for (Num [Char]) arising from a use
      of ‘+’
      • In the expression: 2 + "hola"
      In an equation for ‘it’: it = 2 + "hola"
    \end{minted}
  \end{itemize}
\end{frame}

\begin{frame}[fragile]
  \frametitle{Primeros pasos}
  \framesubtitle{GHCi}
  \begin{itemize}
  \item\texttt{Expresiones útiles:}
    \begin{minted}[bgcolor=bg]{shell}
      Prelude> :t 5
      5 :: Num t => t
      Prelude> :t 2
      2 :: Num t => t
      Prelude> :t "hola"
      "hola" :: [Char]

      Prelude> :l introduccion.hs
      [1 of 1] Compiling Main
      ( introduccion.hs, interpreted )
      Ok, modules loaded: Main.
      *Main>
    \end{minted}
  \end{itemize}
\end{frame}

\subsection{Primeras funciones}
\begin{frame}{Primeros pasos}      %TODO
  \framesubtitle{Primeras funciones}

\end{frame}

\subsection{Listas}
\begin{frame}{Primeros pasos}      %TODO
  \framesubtitle{Listas}

\end{frame}

\subsection{Tuplas}
\begin{frame}{Primeros pasos}      %TODO
  \framesubtitle{Tuplas}

\end{frame}

%%%%%%%%%%%%%%%%%%%%%%%%%%%%%%%%%%%%%%%%%%%%%%%%%%%%%%%%%%%%%%%%%%%%%%%%%%%%% 
% TIPOS Y TYPECLASSES
%%%%%%%%%%%%%%%%%%%%%%%%%%%%%%%%%%%%%%%%%%%%%%%%%%%%%%%%%%%%%%%%%%%%%%%%%%%%% 
\section{Tipos y Typeclasses}
\begin{frame}{Tipos y \emph{Typeclasses}}      %TODO

\end{frame}

%%%%%%%%%%%%%%%%%%%%%%%%%%%%%%%%%%%%%%%%%%%%%%%%%%%%%%%%%%%%%%%%%%%%%%%%%%%%% 
% SINTAXIS EN FUNCIONES
%%%%%%%%%%%%%%%%%%%%%%%%%%%%%%%%%%%%%%%%%%%%%%%%%%%%%%%%%%%%%%%%%%%%%%%%%%%%% 
\section{Sintaxis en funciones}
\begin{frame}{Sintaxis en funciones}      %TODO

\end{frame}

%%%%%%%%%%%%%%%%%%%%%%%%%%%%%%%%%%%%%%%%%%%%%%%%%%%%%%%%%%%%%%%%%%%%%%%%%%%%% 
% RECURSION
%%%%%%%%%%%%%%%%%%%%%%%%%%%%%%%%%%%%%%%%%%%%%%%%%%%%%%%%%%%%%%%%%%%%%%%%%%%%% 
\section{Recursión}
\begin{frame}{Recursión}      %TODO

\end{frame}

\end{document}
